\documentclass[fleqn]{article}[12pt]

\usepackage{amsmath}
\usepackage{amssymb}
\usepackage{pgfplots}
\usepackage[margin=1in]{geometry}
\usepackage{fancyhdr}
\usepackage{lastpage}
\usepackage{siunitx}
\usepackage{amsthm}
\usepackage{booktabs}


\setlength\parindent{0pt}

\cfoot{\thepage \hspace{1pt} / \pageref{LastPage}}
\newcommand{\integral}[4]{\int_#1^#2 \! #3 \, \mathrm{d}#4}
\newcommand{\dif}{\mathrm{d}}
\newcommand{\diracraw}{\left(\int_{-\infty}^{\infty} e^{i2\pi (f - \bar f)}\, dt\right)}
\usepackage{caption}
\captionsetup{justification=raggedright,singlelinecheck=false}
\DeclareMathOperator{\Imag}{Im}


\DeclareSIUnit\year{yr}
\DeclareSIUnit{\calorie}{cal}

\pgfplotsset{compat=1.14}

\newcommand{\M}{\mathbb{M}}
\newcommand{\W}{\mathbb{W}}
\newcommand{\R}{\mathbb{R}}


\begin{document}
    \begin{tabular}{l}
        ID \#33 \\
        Problem Set 13 \\
        Physics 202 \\
        \today
    \end{tabular}

\begin{enumerate}
    \item \begin{enumerate}
        \item As showed last semester, the differential equation for this situation is
        \begin{equation*}
            \ddot{x}(t) = \frac{k}{m}x(t)
        \end{equation*}
        with solution
        \begin{equation*}
            x(t) = x_i\cos(\omega t) = x_i \cos\left(\sqrt{k/m} t\right)
        \end{equation*}
        Then, taking the derivative, the velocity is
        \begin{equation*}
            v_x(t) = -\omega x_i \sin(\omega t)
        \end{equation*}

        \item The probability that the mass can be found at a point $x$ is inversely proportional to the velocity of the mass at that point, since a higher velocity means the mass spends less time at that point. The velocity of the mass at a given point can be found through energy, as follows:
        \begin{equation*}
            E_{\text{tot}} = \frac{1}{2}kx_i^2
        \end{equation*}
        \begin{equation*}
            U(x) = \frac{1}{2}kx^2 \implies K(x) = E_\text{tot} - U(x) = \frac{1}{2}kx_i^2 - \frac{1}{2}kx^2 = \frac{1}{2}k(x_i^2-x^2)
        \end{equation*}
        Then, solving for $v(x)$,
        \begin{equation*}
            \frac{1}{2}m v(x)^2 = \frac{1}{2}k(x_i^2-x^2) \implies v(x) = \sqrt{\frac{k}{m}(x_i^2-x^2)}
        \end{equation*}
        Since the mass cannot be beyond the point $x_i$ (there is not enough energy for it to reach that point),
        \begin{equation*}
            P(x) dx = \begin{cases}
                \frac{A}{\sqrt{x_i^2-x^2}}dx & -x_i\leq x \leq x_i \\
                0 & \text{otherwise}
        \end{cases}
        \end{equation*}

        \item A probability distribution function must be normalized, so we need to find $A$ s.t.
        \begin{equation*}
            \int_{-\infty}^\infty \frac{A}{\sqrt{x_i^2-x^2}}dx = \int_{-x_i}^{x_i} \frac{A}{\sqrt{x_i^2-x^2}}dx = 1
        \end{equation*}
        Evaluating this integral gives that
        \begin{equation*}
            \int_{-x_i}^{x_i} \frac{A}{\sqrt{x_i^2-x^2}}dx = \left. -A \arctan \left(\frac{x \sqrt{x_i^2-x^2}}{x^2-x_i^2}\right)\right|_{x=-x_i}^{x_i} = A \pi \implies A = \frac{1}{\pi}
        \end{equation*}

        \item Since the PDF is symmetric and centered at 0, the block's position has an expected value of $x=0$.
        \begin{equation*}
            \langle x \rangle = \int_{\infty}^{\infty} x P(x) dx = 0
        \end{equation*}

        \item The probability density function for the displacement is given by
        \begin{equation*}
            P_{\text{disp}}(x) dx = \begin{cases}
                \frac{2/\pi}{\sqrt{x_i^2-x^2}}dx & 0\leq x\leq x_i \\
                0 & \text{otherwise}
        \end{cases}
        \end{equation*}
        Then the expected value of the displacement is
        \begin{equation*}
            \langle |x| \rangle = \int_0^{x_i} x \frac{2/\pi}{\sqrt{x_i^2-x^2}}dx = \left. -\frac{2 \sqrt{x_i^2-x^2}}{\pi } \right|_{x=0}^{x_i} = \frac{2x_i}{\pi}
        \end{equation*}
    \end{enumerate}

    \item \begin{enumerate}
        \item Deriving the speed distribution for the 3d case is similar to deriving it for the 2d case. Start by imagining a 3D plot of the velocity components. For every gas particle, $v_x \approx v_y \approx v_z$. Then imagine a cylinder $C$ in that space. Because $g(v_x)dv_x \propto dv_x$ (and similar for the other two directions), switching to cylindrical coordinates gives that
        \begin{equation*}
            g(v)dv = \int_C v dv = \int_{\theta = 0}^{2\pi} \int_{z=0}^{v} v \,dv\, dz\, d\theta = 2 \pi v^2 dv
        \end{equation*}
        Then, as shown in class, the speed distribution function must be
        \begin{equation*}
            P(v) dv = \left(\frac{m}{2\pi k_B T}\right)^{\frac{3}{2}} 2 \pi v^2 e^{-\frac{mv^2}{2k_BT}} dv
        \end{equation*}

        \item
        The average speed of an ideal gas particle is
        \begin{equation*}
            \langle v \rangle = \int_{-\infty}^{\infty} v P(v) dv =
            \int_{0}^{\infty} v \left(\frac{m}{2\pi k_B T}\right)^{\frac{3}{2}} 2 \pi v^2 e^{-\frac{mv^2}{2k_BT}} dv =
            \frac{\sqrt{\frac{2}{\pi }} T^2 k_B^2 \left(\frac{m}{T k_B}\right){}^{3/2}}{m^2}
        \end{equation*}

        \item The distribution function of $v^2$ needs to be multiplied by the constant
        \begin{equation*}
            \frac{8 \sqrt{\pi } T^3 k_B^3 \left(\frac{m}{T k_B}\right){}^{5/2}}{3 m^3}
        \end{equation*}
        to be normalized. Then
        \begin{equation*}
            \langle v^2 \rangle = \int_{-\infty}^{\infty} \frac{8 \sqrt{\pi } T^3 k_B^3 \left(\frac{m}{T k_B}\right){}^{5/2}}{3 m^3} \left(\frac{\sqrt{\frac{2}{\pi }} T^2 k_B^2 \left(\frac{m}{T k_B}\right){}^{3/2}}{m^2}\right)^2 =
            \frac{4 T^3 k_B^3 \left(\frac{m}{T k_B}\right){}^{5/2}}{3 \sqrt{\pi } m^3}
        \end{equation*}
        and
        \begin{equation*}
            \sqrt{\langle v^2 \rangle} = \frac{2}{\sqrt{3} \sqrt[4]{\pi } \sqrt[4]{\frac{m}{T k_B}}}
        \end{equation*}
    \end{enumerate}
\end{enumerate}


\end{document}
