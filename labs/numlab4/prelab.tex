\documentclass{article}

\usepackage{amsmath}

\begin{document}
    \begin{enumerate}
        \item[3.-1] Setting $x$ and $y$ to 0 gives that
        \begin{equation*}
            \vec{B}(\vec{r}) = \frac{\mu_0 I R}{4\pi} \int_{0}^{2\pi} \frac{z \cos \phi \hat{x} + z\sin \phi \hat{y} + R \hat{z}}{\left(R^2\cos^2\phi + R^2 \sin^2 \phi + z^2\right)^{3/2}} d\phi
        \end{equation*}
        However, because $\int_{0}^{2\pi} \cos \theta d\theta = 0$ (and similarly for $\sin$),
        \begin{equation*}
            \vec{B}(\vec{r}) = \frac{\mu_0 I R}{4\pi} \int_{0}^{2\pi} \frac{R \hat{z}}{\left(R^2 + z^2\right)^{3/2}} d\phi
            = \frac{\mu_0 I R}{4\pi} \frac{2\pi R}{(R^2+z^2)^{3/2}} =
            \frac{\mu_0 I R^2}{2(R^2+z^2)^{3/2}}
        \end{equation*}

        \item[3.0] 1.23 states
        \begin{equation*}
            I \approx \sum_{j=0}^{n-1} \frac{1}{2} (F_j+F_{j+1}) \Delta x \equiv I_t
        \end{equation*}
        and 1.19 states that
        \begin{equation*}
            I \approx \sum_{j=0}^{n-1} F_j \Delta x \equiv I_b
        \end{equation*}
        We can see that the first term of the trapezoidal approximation is
        \begin{equation*}
            I_{t,1} = \frac{1}{2} (F_1 + F_2) \Delta_x
        \end{equation*}
        and the second is
        \begin{equation*}
            I_{t,2} = \frac{1}{2} (F_2 + F_3) \Delta_x
        \end{equation*}
        with their sum being
        \begin{equation*}
            I_{t,1+2} = \frac{1}{2}F_1 + F_2 + \frac{1}{2}F_3
        \end{equation*}
        It follows that
        \begin{equation*}
            I_t = I_b - \frac{1}{2}F_1 + \frac{1}{2}F_n
        \end{equation*}
    \end{enumerate}
\end{document}
