\documentclass{article}

\usepackage[fleqn]{amsmath}

\begin{document}
    \begin{enumerate}
        \item A simple function we can work with is the damped harmonic oscillator:
        \begin{equation*}
            m \ddot{x}(t) = -kx(t) - 2mb\dot{x}(t)
        \end{equation*}
        \item Using the Taylor expansion from the writeup above, we have that
        \begin{equation*}
            x_{j+1} = 2x_{j} - x_{j-1} + \Delta t^2 \frac{F_j}{m}
        \end{equation*}
        Then
        \begin{equation*}
            v_j = \frac{x_{j+1} - x_j}{\Delta t} = \frac{x_{j} - x_{j-1} + \Delta t^2 \frac{F_j}{m}}{\Delta t} =
            \frac{x_j - x_{j-1}}{\Delta t} + \Delta t \frac{F_j}{m}
        \end{equation*}

        \item The angular frequency for an oscillator like this is
        \begin{equation*}
            \omega = \sqrt{k/m}
        \end{equation*}
        so the regular frequency is
        \begin{equation*}
            f = \frac{\sqrt{k/m}}{2\pi} \implies T = \frac{2\pi}{\sqrt{k/m}}
        \end{equation*}

        The potential energy of the mass at maximum extension is
        \begin{equation*}
            U = \frac{1}{2}k a^2
        \end{equation*}
        so the maximum speed is
        \begin{equation*}
            v_{\text{max}} = a\sqrt{k/m}
        \end{equation*}
        Therefore, for
        \begin{equation*}
            v_{\text{max}} = c \implies a = c\sqrt{m/k}
        \end{equation*}
    \end{enumerate}
\end{document}
