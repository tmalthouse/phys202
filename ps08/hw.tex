\documentclass[fleqn]{article}[12pt]

\usepackage{amsmath}
\usepackage{amssymb}
\usepackage{pgfplots}
\usepackage[margin=1in]{geometry}
\usepackage{fancyhdr}
\usepackage{lastpage}
\usepackage{siunitx}
\usepackage{amsthm}
\usepackage{booktabs}


\setlength\parindent{0pt}

\cfoot{\thepage \hspace{1pt} / \pageref{LastPage}}
\newcommand{\integral}[4]{\int_#1^#2 \! #3 \, \mathrm{d}#4}
\newcommand{\dif}{\mathrm{d}}
\newcommand{\diracraw}{\left(\int_{-\infty}^{\infty} e^{i2\pi (f - \bar f)}\, dt\right)}
\usepackage{caption}
\captionsetup{justification=raggedright,singlelinecheck=false}
\DeclareMathOperator{\Imag}{Im}


\DeclareSIUnit\year{yr}
\DeclareSIUnit{\calorie}{cal}

\pgfplotsset{compat=1.14}

\newcommand{\M}{\mathbb{M}}
\newcommand{\W}{\mathbb{W}}
\newcommand{\R}{\mathbb{R}}


\begin{document}
    \begin{tabular}{l}
        ID \#33 \\
        Problem Set 8 \\
        Physics 202 \\
        \today
    \end{tabular}

\begin{enumerate}
    \item 1 mol of gold weighs $\SI{197}{\g}$, so 1 gram of gold is $1/197 = \SI{0.00508}{\mol}$. The the total number of gold atoms we need to produce is
    \begin{equation*}
        \SI{0.00508}{\mol} \cdot \SI{6.022e23}{\mol^{-1}} = 3.057\times10^{21}
    \end{equation*}
    Then
    \begin{equation*}
        \frac{3.057\times10^{21}}{\SI{20000}{\s^{-1}}} = \SI{1.53e17}{\s} = \SI{4.85}{\giga\year}
    \end{equation*}
    At this rate, it would take about a third of the age of the universe to create a single gram of gold.

    \item Let the volume of the right chamber be 1, and let $R=1$. Then the pressure in the right chamber is
    \begin{equation*}
        P_R = \frac{nRT}{V} = 290
    \end{equation*}
    The chamber on the right is about 5 times larger, so $V\approx 5$. Then
    \begin{equation*}
        P_L = \frac{nRT}{V} = \frac{350}{5} = 70
    \end{equation*}
    The piston will move right when released, because the pressure in the right chamber is far higher.

    \item Let $a = nRT$. Then $\SI{0.3}{m^3}(\SI{200}{\bar}) = a = V(\SI{1}{\bar}) \implies V = \SI{60}{\m^3}$.
    Then the number of balloons that can be filled is $\SI{60}{\m^3}/\SI{0.01}{\m^3} = 6000$.

    \item The heat needed to bring the tea down from $\SI{100}{\celsius}$ to $\SI{65}{\celsius}$ is
    \begin{equation*}
        Q = cmT = (\SI{1}{\calorie/\gram\celsius})(\SI{200}{\g})(\SI{-35}{\celsius}) = \SI{-7000}{\calorie}
    \end{equation*}
    Because the ice will melt after being put into the glass, the enthalpy of fusion must also be considered. Therefore, the energy required to bring 1 gram of ice to $\SI{65}{\celsius}$ is
    \begin{equation*}
        Q = 0.5 * \SI{15}{\celsius} + 79.72 + 1*\SI{65}{\celsius} = \SI{152.2}{\calorie/\g}
    \end{equation*}
    Then
    \begin{equation*}
        \frac{\SI{7000}{\calorie}}{\SI{152.2}{\calorie/\g}} = \SI{45.98}{\gram}
    \end{equation*}
    so about 46 grams of ice are needed to bring the tea to a comfortable temperature (it'll be very watered down, though.)

\end{enumerate}


\end{document}
