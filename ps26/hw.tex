\documentclass[10pt]{article}

\usepackage{siunitx}
\usepackage{amsmath}
\usepackage{booktabs}
\usepackage[margin=1in]{geometry}


\begin{document}
  \begin{tabular}{l}
    Box Num. 33 \\
    Problem Set 26 \\
    \today
  \end{tabular}

  \begin{enumerate}
    \item \begin{enumerate}
        \item The 16 possible sets of quantum numbers for $n=4$ are

        \begin{tabular}{rrr|rrr}
            \toprule
            $n$ & $l$ & $m_l$ & $n$ & $l$ & $m_l$ \\
            4 & 0 & 0 & 4 & 2 & -2 \\
            4 & 1 & 0 & 4 & 3 & 0 \\
            4 & 1 & 1 & 4 & 3 & 1 \\
            4 & 1 & -1 & 4 & 3 & -1 \\
            4 & 2 & 0 & 4 & 3 & 2 \\
            4 & 2 & 1 & 4 & 3 & -2 \\
            4 & 2 & -1 & 4 & 3 & 3 \\
            4 & 2 & 2 & 4 & 3 & -2
        \end{tabular}

        \item \begin{itemize}
            \item The possible values of $l$ when $n=6$ are $0, 1, \dots, 5$.
            \item The possible values of $m_l$ when $l=6$ are $0, 1, \dots 6$.
            \item The smallest $n$ for which $l$ can be 4 is $n=5$.
            \item The smallest possible $l$ that has a $z$-component of $4\hbar$ is $l=4$.
        \end{itemize}
    \end{enumerate}
    \item The radial component of the wavefunction for this set of qunatum numbers is
    \begin{equation*}
        R(r) = \frac{1}{(2a_0)^{3/2}} \left( 2-\frac{r}{a_0} \right) e^{-r/2a_0}
    \end{equation*}
    so the radial probability is given by
    \begin{equation*}
        |R(r)|^2 = \frac{1}{(2a_0)^3} \left( 2 - \frac{r}{a_0} \right)^2 e^{-r/{a_0}}
    \end{equation*}
    Differentiating and solving for roots gives that there are extrema at $r=2a_0$ and $r=4a_0$. Plugging these into the original function gives that $R(2a_0) = 0$ and $R(4a_0) = -1/\sqrt{2} a_0^{3/2}e^2$, so the maximum is at $r=4a_0$.

    \item For the 1s state of hydrogen,
    \begin{align*}
        r_{\text{av}} &= \int_0^\infty r P(r) dr = \int_0^\infty r^3 |R(r)|^2 dr =
        \int_0^\infty r^3 \left(\frac{2}{a_0^{3/2}}e^{-r/a_0}\right)^2 dr
        = \int_0^\infty r^3 \frac{4}{a_0^3}e^{-2r/a_0} dr = \frac{3}{2} a_0
      \end{align*}
      This is larger than the Bohr radius because the `long tail' of the decaying exponential term means there is a small probability of the electron being found at large radii, which then drag the overall mean upwards.

      \item The average potential energy in this situation is
      \begin{equation*}
          U_{\text{av}} = \int_0^\infty U(r)P(r) = \int_0^\infty -\frac{e^2}{\pi\epsilon_0a_0^3} r \exp(-2r/a_0) dr =
          -\frac{e^2}{4a_0\pi\epsilon_0} = \SI{-27.213}{\electronvolt}
      \end{equation*}
      This is larger than the Bohr model prediction for $n=1$, which predicted an average potential of $-\SI{13.6}{\electronvolt}$.
  \end{enumerate}
\end{document}
