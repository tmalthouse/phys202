\documentclass[11pt]{article}

\usepackage{siunitx}
\usepackage{amsmath}
\usepackage[margin=1in]{geometry}


\begin{document}
  \begin{tabular}{l}
    Box Num. 33 \\
    Problem Set 21 \\
    \today
  \end{tabular}

  \begin{enumerate}
    \item The wavefunction for the ground state of the simple harmonic oscillator is the Gaussian, normalized by the constant $A$:
    \begin{equation*}
      \psi(x) = Ae^{-(\sqrt{km}/2\hbar)x^2}
    \end{equation*}
    Since the Gaussian is symmetric, $x_\text{av}$ is 0. The average squared distance, $(x_{av})^2$, is given by the standard deviation of the Gaussian, as described below. Therefore,
    \begin{equation*}
        \sigma^2 = \frac{\sqrt{km}}{\hbar}
    \end{equation*}

    From section 4.4 of Krane, the positional uncertainty of a Gaussian distribution like this is
    \begin{equation*}
        \Delta x = \sigma_x = \sqrt{(x^2)_{av}-(x_{av})^2}
    \end{equation*}

    \item The simple harmonic oscillator system has a potential energy function of $U=\frac{1}{2} k x^2$, so the Schr\"odinger equation for this system is (from Krane)
    \begin{equation*}
      -\frac{\hbar^2}{2m} \frac{d^2 \psi}{dx^2}
      + \frac{1}{2} k x^2 \psi
      = E \psi
    \end{equation*}
    Then, since we know $\psi$ is of the form $\psi(x) = Axe^{-ax^2}$,
    \begin{equation*}
      \frac{d\psi}{dx} = Ae^{-ax^2} - 2Aax^2e^{-ax^2}
      \text{ and }
      \frac{d^2\psi}{dx^2} = 4a^2 A e^{-a x^2} x^3 - 6 a A e^{-a x^2} x =
      Ae^{-a x^2} \left(4a^2x^3 - 6ax\right)
    \end{equation*}
    Then subbing this derivative into the Schr\"odinger equation gives
    \begin{align*}
      &-\frac{\hbar^2}{2m} Ae^{-a x^2} \left(4a^2x^3 - 6ax\right)
      + \frac{1}{2} k x^2 A x e^{-a x^2}
      = E A x e^{-a x^2} \\
      \implies
      &-\frac{\hbar^2}{2m} \left(4a^2x^3 - 6ax\right)
      +\frac{1}{2} k x^3
      =Ex \\
      \implies
      & \frac{1}{2}kx^2 + \frac{3 a \hbar^2}{m} - \frac{2 a^2 x^2 \hbar^2}{m} = E
      \\
      \intertext{The only way for an equation like this to be true is for the coefficients of $x^2$ to sum to 0, and for the constant coefficients to do the same. This gives that} \\
      &\frac{1}{2} k = \frac{2 a^2 \hbar^2}{m} \text{ and } E = \frac{3 a \hbar^2}{m} \\
      & a = \frac{\sqrt{km}}{2\hbar} \text{ and } E = \frac{3}{2} \hbar \omega_0
    \end{align*}

    \item Since the two points are at the same temperature, and change in entropy is path-independent, this transformation is equivalent to an isotherm. We know that the change in entropy along this isotherm is
    \begin{equation*}
      \Delta S = N k_B \log \frac{V_B}{V_A}
    \end{equation*}
  \end{enumerate}
\end{document}
