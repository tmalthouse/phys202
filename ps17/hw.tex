\documentclass[fleqn]{article}[12pt]

\usepackage{amsmath}
\usepackage{amssymb}
\usepackage{pgfplots}
\usepackage[margin=0.75in]{geometry}
\usepackage{fancyhdr}
\usepackage{lastpage}
\usepackage{siunitx}
\usepackage{amsthm}
\usepackage{booktabs}


\setlength\parindent{0pt}

\cfoot{\thepage \hspace{1pt} / \pageref{LastPage}}
\newcommand{\integral}[4]{\int_#1^#2 \! #3 \, \mathrm{d}#4}
\newcommand{\dif}{\mathrm{d}}
\newcommand{\diracraw}{\left(\int_{-\infty}^{\infty} e^{i2\pi (f - \bar f)}\, dt\right)}
\usepackage{caption}
\captionsetup{justification=raggedright,singlelinecheck=false}
\DeclareMathOperator{\Imag}{Im}


\DeclareSIUnit\year{yr}
\DeclareSIUnit{\calorie}{cal}

\pgfplotsset{compat=1.14}

\newcommand{\M}{\mathbb{M}}
\newcommand{\W}{\mathbb{W}}
\newcommand{\R}{\mathbb{R}}


\begin{document}
    \begin{tabular}{l}
        ID \#33 \\
        Problem Set 17 \\
        Physics 202 \\
        \today
    \end{tabular}

\begin{enumerate}
    \item \begin{enumerate}
        \item These neutrons' kinetic energy is
        \begin{equation*}
            K = \frac{3}{2}kT = \frac{3}{2} (\SI{1.38e-23}{\joule/\kelvin})(\SI{293}{\kelvin}) = \SI{6.06e-21}{\joule} = \SI{0.038}{\eV}
        \end{equation*}

        \item To calculate the momentum of one of these neutrons, we need its velocity. This is given by
        \begin{equation*}
            v = \sqrt{\frac{2K}{m}} = \sqrt{\frac{2(\SI{6.06e-21}{\joule})}{\SI{1.67e-27}{\kg}}} = \SI{2695}{\m/\s}
        \end{equation*}
        Then the momentum is
        \begin{equation}
            p = mv = (\SI{1.67e-27}{\kg})(\SI{2695}{\m/\s}) = \SI{7.20e-24}{\kg\m/\s}
        \end{equation}
        and the de Broglie wavelength is
        \begin{equation*}
            \lambda = \frac{h}{p} = \frac{\SI{6.626e-34}{\joule\s}}{\SI{7.20e-24}{\kg\m/\s}} = \SI{9.21e-11}{\meter} = \SI{92.08}{\pico\meter}
        \end{equation*}
    \end{enumerate}

    \item \begin{enumerate}
        \item Since $K = \frac{1}{2}mv^2 = \frac{5}{2} k T$, and $p = mv$, the momentum
        \begin{equation*}
            p = m\sqrt{\frac{5 k T}{m}} = \sqrt{5 m k T}
        \end{equation*}
        and the de Broglie wavelength is
        \begin{equation*}
            \lambda = \frac{h}{\sqrt{5 m k T}} = \frac{\SI{6.626e-34}{\joule\s}}{\sqrt{5 (\SI{2.3e-24}{\kg})(\SI{1.38e-23}{\joule/\kelvin})(\SI{293}{\kelvin})}} = \SI{1.14e-10}{\meter} = \SI{3.073}{\pico\meter}
        \end{equation*}

        \item In a cubic meter of air, there are $\SI{1.292}{\kg}$ of gas. Assuming all of this gas is $N_2$, which has a molar mass of $\SI{28.03}{\g/\mol}$. Therefore, there are $1.292/0.02803 = \SI{46.09}{\mol/\m^3}$. Each molecule therefore occupies a volume of about
        \begin{equation*}
            V_{\text{molecule}} \approx \frac{\SI{1}{\m^3}}{(\SI{46.09}{\mol/\m^3})(\SI{6.022e23}{/\mol})} = \SI{3.6e-26}{\m^3}
        \end{equation*}
        and so the distance between molecules is about $\sqrt[3]{\SI{3.6e-26}{\m^3}} = \SI{3.3e-9}{\m} = \SI{3.303}{\nano\meter}$. Since this is about 3 orders of magnitude larger than the de Broglie wavelength, quantum effects are not very important at room temperature.

        \item Since the general rule of thumb is that quantum effects become important when the distance between molecules is less than the de Broglie wavelength, we need to find $T$ s.t.
        \begin{equation*}
            \lambda = \frac{h}{p} \approx \SI{3.303}{\nano\meter}
        \end{equation*}
        which can be rewritten as
        \begin{equation*}
            \SI{3.303}{\nano\meter} = \frac{\SI{6.602e-34}{\joule\s}}{\sqrt{5 (\SI{2.3e-24}{\kg}) (\SI{1.38e-23}{\joule/\kelvin}) T}} \implies T = \SI{0.00025}{\kelvin}
        \end{equation*}
    \end{enumerate}

    \item \begin{enumerate}
        \item According to the email, we didn't have to do part (a.)

        \item The rest mass energy of a proton is
        \begin{equation*}
            E = mc^2 = (\SI{1.672e-27}{\kg})(\SI{3e8}{\m/\s})^2 = \SI{1.505e-10}{\joule} = \SI{939.35}{\mega\eV}
        \end{equation*}
        Since the rest mass energy is so much larger than the kinetic energy, we can use classical mechanics. The velocity of the proton in the nucleus is
        \begin{equation*}
            v = \sqrt{\frac{2K}{m}} = \sqrt{\frac{2(\SI{10}{\mega\eV})}{\SI{1.672e-27}{\kg}}} = \SI{4.38e7}{\m/\s}
        \end{equation*}
        Therefore, the de Broglie wavelength is
        \begin{equation*}
            \lambda = \frac{h}{p} = \frac{(\SI{6.626e-34}{\joule\s})}{(\SI{4.38e7}{\m/\s})(\SI{1.672e-27}{\kg})} = \SI{9.05e-15}{\m}
        \end{equation*}
        and
        \begin{equation*}
            \frac{\lambda}{r} = \frac{\SI{9.05e-15}{\m}}{\SI{1e-12}{\m}} = 0.00905
        \end{equation*}
        so wave mechanics are not needed for this problem.

        \item The rest mass energy of an electron is $\SI{510}{\kilo\eV}$, so we can use classical mechanics for this problem. The electrons have a velocity of
        \begin{equation*}
            v = \sqrt{\frac{2K}{m}} = \sqrt{\frac{2(\SI{1.602e-15}{\joule})}{\SI{9.11e-31}{\kg}}} = \SI{5.93e7}{\m/\s}
        \end{equation*}
        and a de Broglie wavelength of
        \begin{equation*}
            \lambda  = \frac{h}{p} = \frac{\SI{6.626e-34}{\joule\s}}{(\SI{5.93e7}{\m/\s})(\SI{9.11e-31}{\kg})} = \SI{1.22e-11}{\meter}
        \end{equation*}
        Even if we wanted the resolution of our screen to be $\SI{1}{\micro\meter}$, we would not have to worry about wave effects. Since real resolutions are far less fine than this, these designers do not have to account for quantum physics.
    \end{enumerate}
\end{enumerate}


\end{document}
