\documentclass[10pt]{article}

\usepackage{siunitx}
\usepackage{amsmath}
\usepackage[margin=1in]{geometry}


\begin{document}
  \begin{tabular}{l}
    Box Num. 33 \\
    Problem Set 23 \\
    \today
  \end{tabular}

  \begin{enumerate}
    \item \begin{enumerate}
        \item The force on the electron in the electric field is given by $F_E = e E$, where $e$ is the charge of the electron, and the time spent in the field is $t=\ell/v$. Therefore, the final $y$-velocity is
        \begin{equation*}
            v_y = \frac{\ell}{v} \frac{e E}{m_e}
        \end{equation*}
        and so the final angle is
        \begin{equation*}
            \theta_E = \arctan \left(\frac{\frac{\ell}{v} \frac{e E}{m_e}}{v}\right) =
            \arctan \left(\frac{\ell e E}{m_e v^2}\right)
        \end{equation*}

        \item The force on the electron in this magnetic field is $F_B = qv\times B$. Because, for circular movement like this,
        \begin{equation*}
            r = \frac{mv^2}{F} = \frac{m v}{e B},
        \end{equation*}
        Then, through setting up a system of similar triangles,
        \begin{equation*}
            \theta_B = \arcsin\left(\frac{\ell e B}{m v}\right)
        \end{equation*}

        \item Using the small-angle approximations for both $\tan$ and $\sin$, when $\theta_B = \theta_E$,
        \begin{equation*}
            \frac{\ell e E}{m_e v^2} \approx \frac{\ell e B}{m_e v} \implies E \approx vB
        \end{equation*}
        \begin{equation*}
            e = \frac{6\pi \eta r (v_1+v_2)}{(1+n)E}
        \end{equation*}
    \end{enumerate}

    \item
    Let the electric field be given by $E = V/d$. Then, in the initial state, we have that
    \begin{equation*}
        m_e g = E e n \implies m_e = \frac{E e n}{g}
    \end{equation*}
    where $n$ is the number of electrons above neutral. Then, after it loses a single electron,
    \begin{equation*}
        E e (n+1) = 6 \pi\eta r v_1 + m_e g
    \end{equation*}
    and finally,
    \begin{equation*}
        m_e g = 6\pi\eta r v_2
    \end{equation*}
    \begin{equation*}
        E e (n+1) = 6 \pi\eta r (v_1+v_2)
    \end{equation*}


    \item \begin{enumerate}
        \item This equation for force is identical to the one for a simple harmonic oscillator, so the angular frequency, $\omega$, is just
        \begin{equation*}
            \omega = \sqrt{k/m} = \sqrt{\frac{e^2}{4\pi\epsilon_0R^3 m_e}} = \SI{4.125}{\radian/\s}
        \end{equation*}
        Then
        \begin{equation*}
            \lambda = \frac{2\pi c}{\omega} = \SI{4.57e-8}{\meter} = \SI{45.7}{\nano\meter}
        \end{equation*}
        This is very different from the strongest emission line in hydrogen.


        \item With sodium,
        \begin{equation*}
            \omega = \SI{6.59e15}{\radian/\s}
        \end{equation*}
        and
        \begin{equation*}
            \lambda = \SI{2.86e-7}{\m} = \SI{286}{\nano\meter}
        \end{equation*}
        This wavelength is also off from the actual value by a significant factor.
    \end{enumerate}
  \end{enumerate}
\end{document}
