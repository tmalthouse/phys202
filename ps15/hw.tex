\documentclass[fleqn]{article}[12pt]

\usepackage{amsmath}
\usepackage{amssymb}
\usepackage{pgfplots}
\usepackage[margin=.75in]{geometry}
\usepackage{fancyhdr}
\usepackage{lastpage}
\usepackage{siunitx}
\usepackage{amsthm}
\usepackage{booktabs}


\setlength\parindent{0pt}

\cfoot{\thepage \hspace{1pt} / \pageref{LastPage}}
\newcommand{\integral}[4]{\int_#1^#2 \! #3 \, \mathrm{d}#4}
\newcommand{\dif}{\mathrm{d}}
\newcommand{\diracraw}{\left(\int_{-\infty}^{\infty} e^{i2\pi (f - \bar f)}\, dt\right)}
\usepackage{caption}
\captionsetup{justification=raggedright,singlelinecheck=false}
\DeclareMathOperator{\Imag}{Im}


\DeclareSIUnit\year{yr}
\DeclareSIUnit{\calorie}{cal}

\pgfplotsset{compat=1.14}

\newcommand{\M}{\mathbb{M}}
\newcommand{\W}{\mathbb{W}}
\newcommand{\R}{\mathbb{R}}


\begin{document}
    \begin{tabular}{l}
        ID \#33 \\
        Problem Set 15 \\
        Physics 202 \\
        \today
    \end{tabular}

\begin{enumerate}
    \item \begin{enumerate}
        \item Planck's radiation intensity equation gives that
        \begin{align*}
            I(\lambda) &= \frac{2\pi h c^2}{\lambda^5}\frac{1}{e^{hc/\lambda k T} - 1} \\
            &\approx
             \frac{2\pi h c^2}{\lambda^5}\frac{1}{(1 + hc/\lambda k T) - 1} \\
            &= \frac{2\pi h c^2}{\lambda^5}\frac{1}{hc/\lambda k T} \\
            &= \frac{2\pi h c^2\lambda k T}{\lambda^5 hc} \\
             &= \frac{2\pi c k T}{\lambda^4}
        \end{align*}
        which is the equation from classical wave theory. Therefore, Planck's theory is consistent with the observed results at small frequencies.

        \item As the frequency increases ($\lambda\to0$), in the classical model, the intensity approaches infinity, since
        \begin{equation*}
            \lim_{\lambda \to 0} \frac{2\pi c k T}{\lambda^4} = \lim_{\lambda \to 0} \frac{1}{\lambda^4} = \infty
        \end{equation*}

        However, for Planck's theory, as $\lambda \to 0$, the exponential term $e^{hc/\lambda k T}-1$ grows to infinity (exponentially, surprisingly) while the polynomial term $\lambda^4$ goes to 0. Since exponential growth always dominates polynomial growth,
        \begin{equation*}
            \lim_{\lambda\to 0} \frac{2\pi h c^2}{\lambda^5}\frac{1}{e^{hc/\lambda k T} - 1} = \lim_{\lambda\to 0} \frac{1}{\infty} = 0
        \end{equation*}

    \end{enumerate}

    \item The derivative of $I(\lambda)$ is
    \begin{align*}
        \frac{dI(\lambda)}{d\lambda} &= \frac{2\pi h c^2}{\lambda^5}\frac{1}{e^{hc/\lambda k T} - 1} \frac{d}{d\lambda} \\
        &= \frac{2\pi h c^2}{\lambda^5}\left(\frac{1}{e^{hc/\lambda k T} - 1}\frac{d}{d\lambda}\right) +
        \frac{1}{e^{hc/\lambda k T} - 1} \left(\frac{2\pi h c^2}{\lambda^5}\frac{d}{d\lambda}\right) \\
        &= \frac{2 \pi  c^3 h^2 e^{-\frac{c h}{k \lambda  T}}}{k \lambda ^7 T}-\frac{10 \pi  c^2 h e^{-\frac{c h}{k \lambda  T}}}{\lambda ^6}
    \end{align*}
    which gives a maximum at
    \begin{equation*}
        \lambda = \frac{c h}{5 k T} \implies \lambda T = \frac{c h}{5 k} = \SI{2.72e-3}{\m\kelvin}
    \end{equation*}
    which is very close to the expected value of $\SI{2.89e-3}{\m\kelvin}$.

    \item Integrating equation  3.41 gives
    \begin{align*}
        \int_0^\infty I(\lambda) &= \int \frac{2\pi h c^2}{\lambda^5} \frac{1}{e^{hc/\lambda k T} - 1} d \lambda \\
        &= \frac{12 \pi  k^4 T^4}{c^2 h^3}
    \end{align*}

    \item From problem 2, we know the most intense wavelength is at
    \begin{equation*}
        \lambda = \frac{\SI{2.89e-3}{\m\kelvin}}{\SI{6000}{\kelvin}} = \SI{4.817e-7}{\meter} = \SI{482}{\nano\meter}
    \end{equation*}
    This is in the visible light range, near the color green. According to \footnote{wikipedia.org/wiki/Color\_vision}, the human eye is most sensitive at wavelengths near $\SI{555}{\nano\meter}$, so this maximum intensity is slightly below the region of maximum sensitivity (our eyes are still very good at recognizing it, though.)
\end{enumerate}


\end{document}
