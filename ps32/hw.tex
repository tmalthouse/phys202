\documentclass[10pt]{article}

\usepackage{siunitx}
\usepackage{amsmath}
\usepackage{amsfonts}
\usepackage{booktabs}
\usepackage[margin=0.75in]{geometry}
\usepackage{graphicx}

\renewcommand{\vec}{\mathbf}
\newcommand{\R}{\mathbb{R}}


\begin{document}
  \begin{tabular}{l}
    Box Num. 33 \\
    Problem Set 32 \\
    \today
  \end{tabular}

  \begin{enumerate}
    \item \begin{enumerate}
        \item There are $2^N$ ways to arrange this lattice.

        \item Define the fully demagnetized state to be when exactly half the spins are pointed up. The number of microstates that satisfy this criterion is
        \begin{equation*}
            \begin{pmatrix}N \\ N/2 \end{pmatrix} = \frac{N!}{2(N/2)!} = \frac{N!}{((N/2)!)^2}
        \end{equation*}
        Since there are 2 ways for the bar to be fully magnetized, the entropy of this state is
        \begin{equation*}
            S_{m} = k_B \log 2
        \end{equation*}
        and the entropy of the fully demagnetized state is
        \begin{align*}
            S_{d} &= k_B \log \frac{N!}{((N/2)!)^2} = k_B \left(\log N! - \log ((N/2)!)^2 \right) \\
            &= k_B \left(\log N! - 2\log (N/2)!\right) \\
            &\approx k_B \left(
                (N \log N - N) - (N \log N/2 - N)
            \right) \\
            &= k_B N (\log N - \log N/2) \\
            &= k_B N \log 2 N
        \end{align*}
        The difference in entropy is therefore
        \begin{equation*}
            \Delta S = k_B (N \log 2 N - \log 2)
        \end{equation*}
    \end{enumerate}

    \item \begin{enumerate}
        \item Starting with
        \begin{align*}
            \langle E \rangle &= -\frac{d}{d\beta}\log Z \\
            &= -\frac{d}{dT} \frac{dT}{d\beta} \log(z) \\
            &= \frac{d}{dT} \log Z k T^2 \\
            &= 2\log Z kT
        \end{align*}
    \end{enumerate}

    \item The Fermi energy for a particle in this gas is (by definition)
    \begin{equation*}
        E_F = \frac{h^2}{2m}\left(\frac{3N}{8\pi V}\right)^{2/3}
    \end{equation*}
    The mean energy of such a gas is then given by
    \begin{align*}
        \langle E \rangle &= \frac{1}{N} \int_{0}^{\infty} E N(E) dE \\
        &= \frac{3h^3}{16\pi V \sqrt{2} m^{3/2} E_F^{3/2}} \int_{0}^{\infty} EN(E)dE
        \intertext{However, because the Fermi-Dirac distribution function is 1 below $E_F$ and 0 above,}
        &= \frac{3h^3}{16\pi V \sqrt{2} m^{3/2} E_F^{3/2}} \frac{8\pi V \sqrt{2}m^{3/2}}{h^3}
        \int_{0}^{E_f} E^{3/2} = \frac{3}{5}E_F
    \end{align*}

    \item \begin{enumerate}
        \item The total energy of the initial kaon is $E_k = m_kc^2$. Then the total energy of each of the pions must be $E_\pi = E_k/2 = m_kc^2/2$. If the rest mass of each pion is $m_\pi$, its rest energy is $m_\pi c^2$. Then
        \begin{equation*}
            E^2 = (pc)^2 + (mc^2)^2 \implies (m_k c^2/2)^2 = (pc)^2 + (m_\pi c^2)^2 \implies |p| = \frac{1}{2}c\sqrt{m_k^2-4m_\pi^2}
        \end{equation*}
        and, since
        \begin{equation*}
            \frac{v}{c} = \frac{pc}{E},
        \end{equation*}
        \begin{equation*}
            v = \frac{pc^2}{E} = \frac{\sqrt{m_k^2-4m_\pi^2}}{m_k} c
        \end{equation*}

        \item The kaon has a mass of $\SI{497.6}{\mega\electronvolt/c^2}$, and the pion has a mass of $\SI{139.6}{\mega\electronvolt/c^2}$. Plugging these values into the equations above gives
        \begin{align*}
            E_\pi &= \SI{248.8}{\mega\electronvolt} \\
            p_\pi &= \SI{205.9}{\mega\electronvolt/c} \\
            v_\pi &= \SI{0.827}{c}
        \end{align*}
    \end{enumerate}
  \end{enumerate}
\end{document}
